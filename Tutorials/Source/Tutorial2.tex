\documentclass[svgnames, table, smaller, final,letterpaper,twoside,8pt,justify]{beamer}
\usetheme{Boadilla}
\usepackage{listings}
%\usepackage[usenames,dvipsnames]{xcolor}
\usepackage{graphics}
\usepackage{hyperref}
\usepackage{enumitem}
\usepackage{ragged2e}
\justifying
\newcommand{\Cpp}{ %
\lstset{ %
	language=C++,                % the language of the code
	basicstyle=\footnotesize,           % the size of the fonts that are used for the code
	numbers=left,                   % where to put the line-numbers
	numberstyle=\tiny\color{gray},  % the style that is used for the line-numbers
	stepnumber=2,                   % the step between two line-numbers. If it's 1, each line 
	%will be numbered
	numbersep=5pt,                  % how far the line-numbers are from the code
	backgroundcolor=\color{white},      % choose the background color. You must add \usepackage{color}
	showspaces=false,               % show spaces adding particular underscores
	showstringspaces=false,         % underline spaces within strings
	showtabs=false,                 % show tabs within strings adding particular underscores
	frame=single,                   % adds a frame around the code
	rulecolor=\color{black},        % if not set, the frame-color may be changed on line-breaks within not-black text (e.g. commens (green here))
	tabsize=10,                      % sets default tabsize to 2 spaces
	captionpos=b,                   % sets the caption-position to bottom
	breaklines=true,                % sets automatic line breaking
	breakatwhitespace=false,        % sets if automatic breaks should only happen at whitespace
	title=\lstname,                   % show the filename of files included with \lstinputlisting;
	% also try caption instead of title
	keywordstyle=\color{blue},          % keyword style
	commentstyle=\color{dkgreen},       % comment style
	stringstyle=\color{DarkOrchid}         % string literal style
	escapeinside={\%*}{*)}            % if you want to add LaTeX within your code
	%morekeywords={*,...}               % if you want to add more keywords to the set
}
}
\newcommand{\TS}{\lstset{ style=TS }}
\newcommand{\TSS}{\lstset{ style=TSS }}

\lstdefinelanguage{TorqueScript}{
    showstringspaces=false
    sensitive=false,
    keywords=[0]{function, datablock, delete, messageboxok, exec, echo, getcount, commandtoclient, commandtoserver, schedule, if, else, new, getObject, addObject,
    				singleton, bind, getRowNumById, addRow, sortNumerical, clearSelection,setRowById,removeRowById,getWord,setWord,getWordCount,getField,setField,
    				StripMLControlChars},
    keywords=[1]{StaticShapeData, ParticleEmitterData, ParticleEmitterNodeData, ParticleData, ParticleEmitterNode,ClientGroup, SimGroup, SimObject,
    				GuiControl, GuiBitmapBorderControl, GuiBitmapControl, GuiControlProfile, GuiTextListCtrl, GuiScrollCtrl, GuiTextCtrl, GuiPanel},
    keywords=[2]{SPC,@,TAB},
    keywords=[3]{\$CoinsFound},
    keywords=[5]{this},
    morestring=[s][\color{Purple!100}]{"}{"},
    morestring=[s][\color{Purple!100}]{'}{'},
    morecomment=[l][\color{ForestGreen}]{//},
    morecomment=[s][\color{ForestGreen}]{/*}{*/},
    escapeinside={-*}{*)},
}

\lstdefinestyle{TS}{
    language=TorqueScript,
    keywordstyle=[0]{\color{Blue}},
    keywordstyle=[1]{\color{DarkBlue}},
    keywordstyle=[2]{\color{LightGreen}},
    keywordstyle=[3]{\color{Orange!100}},
    keywords=[4]{obj,obj1,obj2,col,vec,len,score,emitterNode,idx,idxclient,winnerclient,guiContent,val,score,kills,text,name,deaths,clientId,clientName,msgType,msgString},
    keywordstyle=[4]{\color{CornflowerBlue!80}\%},
    keywordstyle=[5]{\color{blue}\%},
}
\lstdefinestyle{TSS}{
    language=TorqueScript,
    keywordstyle=[0]{\color{Blue}},
    keywordstyle=[1]{\color{DarkBlue}},
    keywordstyle=[2]{\color{LightGreen}},
    keywordstyle=[3]{\color{Orange!100}},
    keywords=[4]{obj,obj1,obj2,col,vec,len,score,emitterNode,idx,idxclient,winnerclient,guiContent,val,text,name,clientId,clientName,msgType,msgString},
    keywordstyle=[4]{\color{CornflowerBlue!80}\%},
    keywordstyle=[5]{\color{blue}\%},
}


\author{Lukas Peter Joergensen}
\date{\today}

\newlist{longenum}{enumerate}{5}
\setlist[longenum,1]{label=$\circ$}
\setlist[longenum,2]{label=$\circ$}
\setlist[longenum,3]{label=$\circ$}
\setlist[longenum,4]{label=$\circ$}
\setlist[longenum,5]{label=$\circ$}

\title{Beginners guide to T3D part two}
\begin{document}
%First page
\begin{frame}
\titlepage
{\it Revision 1}
\end{frame}

\section{Introduction}
\begin{frame}
\frametitle{Introduction}
	This is part 2 of a beginners introduction to T3D tutorial series I am writing.\\
	In this tutorial we will be extending the coin collection game with a visual effect when you pick up coins, 
	and a quick rewrite of the original code to support muliplayer interaction and an description of what to take in mind when writing games for multiplayer games.\\
\end{frame}
%Author
\begin{frame}[fragile]
\frametitle{The Author}
	\structure{Who am i?}\\
	My name is Lukas Joergensen, I am 19 years old and live in Denmark, currently studying Computer Science at Aarhus University.\\
	\structure{What is your experience with T3D?}\\
	Currently, my most actual project is the IPS Lite and Pro for T3D. You can read about the IPS Lite on the {\color{blue}\href{https://github.com/lukaspj/IPS-Lite-for-T3D}{GitHub page}}\\
	Or visit my personal website at {\color{blue}\href{http://fuzzyvoidstudio.com/}{FuzzyVoidStudio.com}}
\end{frame}
%Agenda
\begin{frame}
\frametitle{Outline}
\tableofcontents[]
\end{frame}

\section{Building on top of the coin collection game}
\subsection*{Dynamic instancing of objects}
\begin{frame}
\frametitle{Visual effects}
{\bf As you may have noticed}, I love particles. So we will work with particles and emitters in this section where I will cover dynamic instancing of objects\\
What we will create is a small explosion of particles when you pick up a coin.\\
First we want to define a new datablock for the particle emitter we will be using later.\\
In {\it art/datablocks} create a new file {\bf CoinDatablock.cs}\\
To execute datablocks you should do it in the datablockExec.cs file in the same folder.
\end{frame}

\begin{frame}[fragile]
\frametitle{CoinDatablock.cs}
\TS
\begin{lstlisting}
datablock ParticleData(CoinParticle : DefaultParticle)
{
};

datablock ParticleEmitterData(CoinEmitter : DefaultEmitter)
{
   particles = DefaultParticle;
};

datablock ParticleEmitterNodeData(CoinNode : DefaultEmitterNodeData)
{
   timeMultiple = 1.0;
};
\end{lstlisting}
As you can see we define two new {\bf datablocks} in this file, a {\bf ParticleEmitterDatablock} for creating a new {\it particle emitter}. {\it Particle emitters}
does not have a place in the world by themselves tho, they only emit particles they need a node to know where to emit particles from.
Therefore we need a datablock for a particle emitter node aswell.\\
What you probably will notice here is when we define the name of the datablocks, we write a colon and then another name. {\bf What does this mean?}\\
This means that our new datablock, inherits values from another datablock. It makes a copy of the other datablock and lets you edit
the values which will only affect the new datablock.\\
{\bf Another thing that is important to note,} is that {\bf CoinEmitter} references {\bf CoinParticle}. Which is why it is important that
{\bf CoinParticle} is defined {\it before} {\bf CoinEmitter}
\end{frame}

\begin{frame}[fragile]
\frametitle{The ParticleEmitter blocks}
You have already seen an example of how a datablock is written. But I wrote the CoinParticle for you anyway:\\
\TS
\begin{lstlisting}
datablock ParticleData(CoinParticle : DefaultParticle)
{
   lifetimeMS = 1000;
   gravityCoefficient = 0;
   dragCoefficient = "2";
   
   sizes[0] = 1;
   sizes[1] = 1;
   sizes[2] = 1;
   sizes[3] = 1;
   inheritedVelFactor = "0";
};
\end{lstlisting}
You can copy and paste that to the CoinDatablock.cs\\
{\bf Now a little task for you,} I want you to create your own emitter. You can either use the 
{\color{blue} \href{http://docs.garagegames.com/torque-3d/official/content/documentation/World\%20Editor/Editors/ParticleEditor.html}{ParticleEditor}}
in the World Editor or try different values by writing them in script and then see how it looks in-game.\\
If you decide to script it in hand, then I wrote a small list of {\color{blue}\hyperlink{QG-pNode}{important ParticleEmitter field values}}.\\
If you think this is a waste of time, you can find my ParticleEmitterDatablock on the next frame. However I can ensure you, it is not a waste of time.
\end{frame}

\begin{frame}[fragile]
\frametitle{My ParticleEmitterDatablock}
\TS
\begin{lstlisting}
datablock ParticleEmitterData(CoinEmitter : DefaultEmitter)
{
   particles = CoinParticle;
   ejectionPeriodMS = "10";
   ejectionVelocity = "4.167";
   ejectionOffset = "0.625";
   thetaMax = "360";
   softnessDistance = "1";
   lifetimeMS = "200";
};
\end{lstlisting}
An important thing to note here is that i set the softnessDistance to 1.\\
It defaults to 1000 so it is important to set this down to something reasonable, or else your particles will look transparent when the background is not kilometres away.
\end{frame}

\begin{frame}[fragile]
\frametitle{Instantiating new emitters on the fly}
Lets put these new datablocks to good use. We want some visual feedback to tell us that we have picked up a coin.\\
To spawn a new Emitter we will use the {\bf new} operator. It works like this:
\TS
\begin{lstlisting}
emitterNode = new ParticleEmitterNode(){
   datablock = CoinNode;
   emitter = CoinEmitter;
};
\end{lstlisting}
Remember it is the node not emitter we want to spawn, then we set the emitter inside the "constructor". What i call the constructor is
the variable definitions inside the two brackets \{ and \}.\\ 
This is where we define what datablock to use, the emitter and anything else we want to do with the newly created object.\\
We need to give this new object a position in the world.\\
To get the position of an object you would call {\bf \%obj.getPosition();}\\
And to set the position of an object you would assign it, like this {\bf \%obj.position = "x y z";}\\
Can you figure out where to spawn the new emitter? And how to set it's position (one answer can be found on next frame)\\
{\it Hint: look in Coin.cs}
\end{frame}

\begin{frame}[fragile]
\frametitle{Spawning emitters when picking up coins}
\structure{This is how I chose to solve the problem.}
\TS
\begin{lstlisting}
function Coin::onCollision(this, obj, col, vec, len)
{
   emitterNode =  new ParticleEmitterNode(){
      datablock = CoinNode;
      emitter = CoinEmitter;
      position = obj.getPosition();
   };
   obj.delete();

   $CoinsFound++;
   if(Coins.getCount() <= 0)
   {
      commandToClient(col.client,'ShowVictory',$CoinsFound);
   }
}
\end{lstlisting}
Simply create the new emitter, and give it the same position as the coin.\\
\end{frame}

\begin{frame}[fragile]
\frametitle{Schedules and cleanup}
If you run into a couple of Coins, and it is all working properly, then if you open the world editor you will notice that the emitters is still there even tho
they stopped emitting particles (given that you gave the ParticleEmitter a lifetime) if you didn't you will see that they keep emitting particles.\\
We want to fix that! So I will introduce you to a very important feature in TorqueScript. {\bf Schedules}. You can use a schedule to delete the emitter
after some time.\\
The schedule syntax is:\\
{\bf \%obj.schedule(float time, string method, string args...);}\\
Or if you are not calling it on an object:\\
{\bf schedule(float time, int objectID, string method, string args...)}\\
We can use this to delete the emitter after we spawn it:
\TS
\begin{lstlisting}
emitterNode =  new ParticleEmitterNode(){
   datablock = CoinNode;
   emitter = CoinEmitter;
   position = obj.getPosition();
};
emitterNode.schedule(200,"delete");
\end{lstlisting}
\end{frame}

\section{Gearing the game for multiplayer}
\begin{frame}
\frametitle{Gearing the game for multiplayer}
So you want to play this brand new game with your friends? Well then there is a few things we need to fix. 
{\bf Unfortunately} I have deliberately made the game non-multiplayer compatible.\\
Now lets see how this can be. You remember the {\bf \$CoinsFound} global which kept track of the player score?\\
This global was defined in the {\bf Coins.cs} which resided in the {\it Scripts/server} directory, and was executed by the scriptExec.cs in the server folder.\\
The Coins.cs is actually set on the server and the onCollision trigger, increasing the score is also happening on the server. So if there were more than one client
they would have the same score!\\
Maybe that is what you want but that is not what I would want for my awesome coin collecting game. So lets change this code.
\end{frame}

\begin{frame}[fragile]
\frametitle{The client group, and dynamic variables}
Now lets begin, first I will tell you something about dynamic variables. TorqueScript has something called {\bf dynamic variables} which is a very simple but very smart concept.\\
Basically you can define any variable on any object by assigning it to a value.
\TS
\begin{lstlisting}
obj.somedynVar = 2;
echo(obj.somedynVar); //outputs "2"
obj.TSisSpecial = "Is it now?";
echo(obj.TSisSpecial); //outputs "Is it now?"
echo(obj.tsisspecial); //outputs "Is it now?"
//(TorqueScript is not case sensitive either.)
\end{lstlisting}
We can use this to keep track on how many coins each client has picked up!
\begin{lstlisting}
col.client.coinsfound++; // Automatically starts at 0
\end{lstlisting}
So what is this ClientGroup I talked about? The ClientGroup is a collection of all the clients who have joined the game. We can use this group to access all the clients and send
global messages or iterate through them to find the winner.
\end{frame}

\begin{frame}
\frametitle{Multiplayer support in the coin collection game}
Lets start with the simple things. A little task. Remember the {\bf Scripts/client/commands.cs}? We need to edit this file and add a {\it clientCmdShowDefeat}.\\
Make this function, use the {\it clientCmdShowVictory} as a template.\\
We also need to edit the Coin.cs file. Change the onCollision function to something like the code on the next frame.
\end{frame}
\begin{frame}[fragile]
\TS
\begin{lstlisting}
   emitterNode =  new ParticleEmitterNode(){
      dataBlock = CoinNode;
      emitter = CoinEmitter;
      position = obj.getPosition();
   };
   emitterNode.schedule(200,"delete");
   obj.delete();
   
   col.client.coinsfound++;
   if(Coins.getCount() <= 0)
   {
      winnerClient = col.client;
      for(idx = 0; idx < ClientGroup.getCount(); idx++)
      {
         idxClient = ClientGroup.getObject(idx);
         if(idxClient.coinsfound > winnerClient.coinsfound){
         	commandToClient(winnerClient,
         	    'ShowDefeat',winnerClient.coinsfound);
            winnerClient = idxClient;}
         else
            commandToClient(idxClient,
                'ShowDefeat',idxClient.coinsfound);
      }
      commandToClient(winnerClient,
          'ShowVictory',idxClient.coinsfound);
   }
\end{lstlisting}
\end{frame}

\begin{frame}
\frametitle{Networking what happened?}
The first thing you should notice is that we swapped the {\bf \$CoinsFound} out with {\it \%col.client.coinsfound} so that now the score is on a per-client basis.\\
Then we changed the wincondition out with a new iterative loop where we find the client with the highest amount of coins found and put him in the first place.\\
All the clients who did not come in first place will get a defeat message. If someone has higher score than the former first place he will get a defeat message.\\
Now your first multiplayer game should actually be working! Try opening 2 instances of Torque, in one of the instances you press {\bf "play"} then you tick the
{\bf "host"} check box to the left of the {\bf Go} button.\\
In the other instance you press join, {\bf "Query LAN"} select the server that comes forth and join the game. Now you can compete with yourself about collecting most coins!\\
Even better, you can host a LAN and let all your friends play your coin collection game with you! Give it a cool name and brag about it a little!\\
\structure{But what can you improve}\\
Use your imagination! I will give you one last task tho, in the above code, there can only be one winner. What happens if two persons have an equal amount of coins collected?\\
Send any suggestions or needs for improvements to {\bf LukasPJ@FuzzyVoidStudio.com}.\\
Also if you have a request for a tutorial covering a specific topic please mail me on that address aswell!
\end{frame}

\begin{frame}
\frametitle{Default Datablocks}
The DefaultDatablocks we inherit our datablocks is simply some datablocks which is defined in the {\bf Core} folder which allows us for fast prototyping of new emitters.
\end{frame}

\begin{frame}
\frametitle{ParticleEmitterNode important field variables}
\hypertarget{QG-pNode}{}
\begin{center}
\rowcolors{1}{LightBlue}{Azure}
\begin{tabular}{r p{10cm}}
\hline
Point3 & alignDirection\\
 & The direction aligned particles should face, only valid if alignParticles is true.\\
\hline
bool & alignParticles\\
 & If true, particles always face along the axis defined by alignDirection. \\
 \hline
float & ejectionOffset\\
 & Distance along ejection Z axis from which to eject particles.\\
 \hline
float & ejectionVelocity\\
 & Particle ejection velocity.\\
 \hline
int & ejectionPeriodMS\\
 &  Time (in milliseconds) between each particle ejection.\\
 \hline
int & periodVarianceMS\\
 & Variance in ejectionPeriod. Should never be less than 1 or higher than ejectionPeriodMS\\
 \hline
int & lifetimeMS\\
 & Lifetime of emitted particles (in milliseconds).\\
 \hline
int & lifetimeVariance\\
 & Variance in particle lifetime. Should never be less than 0 or higher than lifetimeMS\\
 \hline
string & particles\\
 & List of space or TAB delimited ParticleData datablock names.\\
 \hline
float & phiReferenceVel\\
 & Reference angle, from the vertical plane, to eject particles from.\\
 \hline
float & phiVariance\\
 & Variance from reference angle, from 0 - 360\\
 \hline
float & thetaMin\\
 & Minimum angle, from the horizontal plane, to eject from.\\
 \hline
float & thetaMax\\
 & Maximum angle, from the horizontal plane, to eject from.\\
 \hline
float & velocityVarianca\\
 & Variance for ejection velocity, from 0 - ejectionVelocity. \\
 \hline
float & softnessDistance\\
 & For soft particles, the distance (in meters) where particles will be faded based on the difference in depth between the particle and the scene geometry. \\
 \hline
\end{tabular}
\end{center}
\end{frame}

\begin{frame}
\frametitle{ParticleEmitterNode important field variables (continued)}
\begin{center}
\rowcolors{1}{LightBlue}{Azure}
\begin{tabular}{r p{10cm}}
float & thetaMax\\
 & Maximum angle, from the horizontal plane, to eject from.\\
 \hline
float & velocityVarianca\\
 & Variance for ejection velocity, from 0 - ejectionVelocity. \\
 \hline
float & softnessDistance\\
 & For soft particles, the distance (in meters) where particles will be faded based on the difference in depth between the particle and the scene geometry. \\
\end{tabular}
\end{center}
\end{frame}

\end{document}